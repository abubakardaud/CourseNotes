\documentclass[10pt, a4paper]{article}
\usepackage[utf8]{inputenc}
\usepackage{amsmath}
\usepackage{amssymb}
\usepackage{dsfont}
\usepackage{xcolor}
\usepackage{amsthm}
\usepackage{graphicx}
\usepackage{xcolor}

\pagecolor[RGB]{47,52,63} %black
\color[RGB]{236,239,244}
\title{Winter Notes}
\author{Abu bakar Daud }
\date{01 Jan 2021 }


\begin{document}
\maketitle 

\section{Cauchy}

Prove that every Cauchy sequence is bounded $(a_n)^{\infty}_{n=1}  $
\begin{proof}


if $a_n$ is a Cauchy sequence then we know that eventually be $\epsilon$ steady, by definition of Cauchy there exists an element N such that for every m,n > N there exists an $\epsilon$ such that
$|m-n| < \epsilon$ we then can split the sequence into a finite part and an infinite part, notably, ${0,N}$ and ${N,\infty}$ 
we know that the infinite sequence will be bounded by any $n,m + \epsilon$ to obtain some upper bound. 


\subsection*{remark}
Note that all the numbers defined in the sequence must be well defined. There cannot be inf or anything like 1/0 otherwise the sequence defintion breaks down as it must exist in the reals. 



\end{proof}




\section*{A8}

Let $L \subseteq \mathbb{R}  $ the set L is called open if for any $x \in L$ there exists $\epsilon > 0$ such that $(x - \epsilon, x + c   ) \subseteq L$ the set is called closed if it's the complement is not open.   


define open interval, we know that an open interval is an interval that does not include it's open points noted by ]a,b[ or (a,b). Closed means it includes the end points [a,b] 



\begin{proof}
	
\end{proof}







\end{document}


